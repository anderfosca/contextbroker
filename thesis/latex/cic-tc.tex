% 
% exemplo genérico de uso da classe iiufrgs.cls
% $Id: iiufrgs.tex,v 1.1.1.1 2005/01/18 23:54:42 avila Exp $
% 
% This is an example file and is hereby explicitly put in the
% public domain.
% 
\documentclass[cic,tc,english]{iiufrgs}
% Para usar o modelo, deve-se informar o programa e o tipo de documento.
% Programas :
% * cic       -- Graduação em Ciência da Computação
% * ecp       -- Graduação em Ciência da Computação
% * ppgc      -- Programa de Pós Graduação em Computação
% * pgmigro   -- Programa de Pós Graduação em Microeletrônica
% 
% Tipos de Documento:
% * tc                -- Trabalhos de Conclusão (apenas cic e ecp)
% * diss ou mestrado  -- Dissertações de Mestrado (ppgc e pgmicro)
% * tese ou doutorado -- Teses de Doutorado (ppgc e pgmicro)
% * ti                -- Trabalho Individual (ppgc e pgmicro)
% 
% Outras Opções:
% * english    -- para textos em inglês
% * openright  -- Força início de capítulos em páginas ímpares (padrão da
% biblioteca)
% * oneside    -- Desliga frente-e-verso
% * nominatalocal -- Lê os dados da nominata do arquivo nominatalocal.def


% Use unicode
\usepackage[utf8]{inputenc}   % pacote para acentuação

% Necessário para incluir figuras
\usepackage{graphicx}         % pacote para importar figuras

\usepackage{listings}

\usepackage{times}            % pacote para usar fonte Adobe Times
% \usepackage{palatino}
% \usepackage{mathptmx}       % p/ usar fonte Adobe Times nas fórmulas

\usepackage[alf,abnt-emphasize=bf]{abntex2cite}	% pacote para usar citações abnt
\newcommand\todo[1]{\textcolor{red}{#1}}

%
% Define o caminho (subpasta) onde você colocará os arquivos de imagem a serem incluídos no texto final
%
\graphicspath{{figures/}} % se tirar fora este comando, por padrão usa a pasta atual

% 
% Informações gerais
% 
\title{Development of a Context Broker with High Availability resource}

\author{Foscarini}{Anderson Didoné}
% alguns documentos podem ter varios autores:
% \author{Flaumann}{Frida Gutenberg}
% \author{Flaumann}{Klaus Gutenberg}

% orientador e co-orientador são opcionais (não diga isso pra eles :))
\advisor[Prof.~Dr.]{Weber}{Taisy Silva}
\coadvisor[M.Sc.]{Crippa}{Marcos Rates}

% a data deve ser a da defesa; se nao especificada, são gerados
% mes e ano correntes
\date{July}{2015}

% o local de realização do trabalho pode ser especificado (ex. para TCs)
% com o comando \location:
\location{Porto Alegre}{RS}

% itens individuais da nominata podem ser redefinidos com os comandos
% abaixo:
% \renewcommand{\nominataReit}{Prof\textsuperscript{a}.~Wrana Maria Panizzi}
% \renewcommand{\nominataReitname}{Reitora}
% \renewcommand{\nominataPRE}{Prof.~Jos{\'e} Carlos Ferraz Hennemann}
% \renewcommand{\nominataPREname}{Pr{\'o}-Reitor de Ensino}
% \renewcommand{\nominataPRAPG}{Prof\textsuperscript{a}.~Joc{\'e}lia Grazia}
% \renewcommand{\nominataPRAPGname}{Pr{\'o}-Reitora Adjunta de P{\'o}s-Gradua{\c{c}}{\~a}o}
% \renewcommand{\nominataDir}{Prof.~Philippe Olivier Alexandre Navaux}
% \renewcommand{\nominataDirname}{Diretor do Instituto de Inform{\'a}tica}
% \renewcommand{\nominataCoord}{Prof.~Carlos Alberto Heuser}
% \renewcommand{\nominataCoordname}{Coordenador do PPGC}
% \renewcommand{\nominataBibchefe}{Beatriz Regina Bastos Haro}
% \renewcommand{\nominataBibchefename}{Bibliotec{\'a}ria-chefe do Instituto de Inform{\'a}tica}
% \renewcommand{\nominataChefeINA}{Prof.~Jos{\'e} Valdeni de Lima}
% \renewcommand{\nominataChefeINAname}{Chefe do \deptINA}
% \renewcommand{\nominataChefeINT}{Prof.~Leila Ribeiro}
% \renewcommand{\nominataChefeINTname}{Chefe do \deptINT}

% A seguir são apresentados comandos específicos para alguns
% tipos de documentos.

% Relatório de Pesquisa [rp]:
% \rp{123}             % numero do rp
% \financ{CNPq, CAPES} % orgaos financiadores

% Trabalho Individual [ti]:
% \ti{123}     % numero do TI
% \ti[II]{456} % no caso de ser o segundo TI

% Monografias de Especialização [espec]:
% \espec{Redes e Sistemas Distribuídos}      % nome do curso
% \coord[Profa.~Dra.]{Weber}{Taisy da Silva} % coordenador do curso
% \dept{INA}                                 % departamento relacionado

% 
% palavras-chave
% iniciar todas com letras minúsculas, exceto no caso de abreviaturas
% 
\keyword{high availability}
\keyword{context broker}
\keyword{context computing}
\keyword{fault tolerance}

%\settowidth{\seclen}{1.10~}

% 
% inicio do documento
% 
\begin{document}

% folha de rosto
% às vezes é necessário redefinir algum comando logo antes de produzir
% a folha de rosto:
% \renewcommand{\coordname}{Coordenadora do Curso}
\maketitle

% dedicatoria
 \clearpage
 \begin{flushright}
     \mbox{}\vfill
     {\sffamily\itshape
       ``The true sign of intelligence is not knowledge,\\ 
       but imagination.''\\}
     --- \textsc{Albert Einstein}
 \end{flushright}

% agradecimentos
\chapter*{Acknowledgements}
I would like to express my very great appreciation to both my advisors, for the patience and kindness on the journey of getting this work done.

Special thanks to my parents, for all the support they gave me throughout the years, always providing the best conditions for my living. I surely know how lucky I am, and I value it every day. I wouldn't be here if it wasn't for you.

I am particularly grateful for the good moments and the assistance given by my brother during the years, both computer and life related, specially in the final days of this work.

I must save a little space to thank all my friends, the old and the new ones, who made the last 7 years of my life very pleasant. A special thanks to those who picked up on me so I would focus and finish this work.

I would like to thank UFRGS and the Informatics Institute, for the excellence on learning and research, being the best computer science course in the country for many years, and having international recognition.. This is where I met great professors from whom I learned not only the theory and application of computer science, but also lessons in life. It was surely a great honor to grow both academically and personally inside this great atmosphere, and if I ever decide to continue my studies, this is a p

And finally thank you, for reading this. I hope you have a good time.

% resumo na língua do documento
\begin{abstract}
    
    \emph{Context is everywhere}. From a mobile phone in a person's pocket to a temperature sensor in the middle of a forest. Context data can be used to help determine the importance of information and services on an environment, and decide whether to make them available to users or not, and how to process them. Applications that use context are called context-aware. With today's large offer of sensing technologies, and the presence of various kinds of sensors in every mobile phone, it has become easier and more useful to sense context in several situations. The use of context data has become a very powerful tool of personalizing interaction with users and the behavior of systems. By improving the computer's access to context, the richness of communication in human-computer interaction and the presence of more useful computational services can be increased.
    
    Context-aware computing is a new and broad area of research. Although its wide range of applicability, not many efforts are made towards specific solutions, including ones related to dependability and fault tolerance. 
    
    This work's main objective is to develop a Context Broker following a previous definition, on a different platform, and to propose a way to add a resource that gives the Broker High Availability characteristics, taking a first step towards the development of a Dependable Context Broker.
\end{abstract}

% resumo na outra língua
% como parametros devem ser passados o titulo e as palavras-chave
% na outra língua, separadas por vírgulas
\begin{englishabstract}{Desenvolvimento de Broker de Contexto com Recurso de Alta Disponibilidade}{Alta disponibilidade. broker de contexto. computação de contexto. tolerância a Falhas}
    \emph{Contexto está em todo lugar}. De um telefone celular no bolso de uma pessoa à um sensor de temperatura no meio de uma floresta. Dados de contexto podem ser usados para auxiliar na definição da importância de informações e serviços em um ambiente, e decidir se as torna disponíveis ao usuário ou não, e como pode processá-las. Aplicações que usam contexto são chamadas sensíveis ao contexto. Com a atual grande oferta de tecnologias sensoriais, e a presença de vários tipos de sensores em cada telefone celular, têm se tornado mais fácil e prático captar contexto em diversas situações. O uso de informações de contexto se tornou uma ferramenta poderosa para personalizar a interação com usuários e o comportamento de sistemas. Ao melhorar-se o acesso de um computador ao contexto, aumenta-se a riqueza das comunicações em interações humano-computador e a presença de serviços computacionais mais úteis.
    
    Computação sensível a contexto é uma nova e ampla área de pesquisa. Não são muitos os esforços em soluções específicas, inclusive as relacionadas à área de tolerância a falhas e confiabilidade.
    Este trabalho tem como objetivo desenvolver um Broker de Contexto e propor a adição e recursos que o dêem características de alta disponibilidade, dando um primeiro passo em direção ao desenvolvimento de um Broker de Contexto  Confiável (Dependable).
\end{englishabstract}

% lista de figuras
\listoffigures

% lista de tabelas
%\listoftables

% lista de abreviaturas e siglas
% o parametro deve ser a abreviatura mais longa
\begin{listofabbrv}{SPMD}
    \item[CxB] Context Broker
    \item[CxC] Context Consumer
    \item[CxP] Context Provider
    \item[REST] Representational State Transfer
    \item[HTTP] Hyper-Text Transfer Protocol
    \item[UFRGS] Universidade Federal do Rio Grande do Sul
    \item[TU-KL] Technische Universität Kaiserslautern
    
\end{listofabbrv}

% idem para a lista de símbolos
% \begin{listofsymbols}{$\alpha\beta\pi\omega$}
%     \item[$\sum{\frac{a}{b}}$] Somatório do produtório
%     \item[$\alpha\beta\pi\omega$] Fator de inconstância do resultado
% \end{listofsymbols}

% sumario
\tableofcontents

% aqui comeca o texto propriamente dito

% inclusion of the chapters
\chapter{Introduction}
Bla \cite{arsanjani2004service} \cite{avivzienis2004basic} \cite{baldauf2007survey} \cite{barreradesign} \cite{dey2000providing} \cite{kian2010federated} \cite{knappmeyer2010contextml} \cite{skeen1981nonblocking} \cite{crippa2010}

\begin{description}
\item[Motivation]content...


\item[Goals]content...


\end{description}

The text is organized as follows: 
\chapter{Definitions}
\label{chap:definitions}
This work proposes the application of a high availability technique to a context broker system. For a better understanding of the system and its development, definitions of Context, Context-Aware System, Context Representation, Fault Tolerance and High Availability related terms are presented.

\section{Context}
\label{sec:context}

Context has had many definitions throughout the years. The first time Context was defined regarding the human-computer interaction was by \cite{schilit1994disseminating}, where context is defined related to location, identities of nearby people and objects, and the changes happening to those.  Similarly, \cite{brown1997context} define context as location, people around the user, time of day, season, temperature, etc. Many authors have also defined context using synonyms, as \cite{brown1995stick} had the idea of “environment”, i.e., what the computer knows about the user's environment, or \cite{franklin1998all} that see context as user's situation. Thus, a lot of definitions have existed, but they all end up being too specific. Context is about the whole situation of an application and its users, and we can't really.

\cite{dey2000providing} defined context as follows: ''Context is any information that can be used to characterize the situation of an entity. An entity is a person, place, or object that is considered relevant to the interaction between a user and an application, including the user and application themselves.'' This is the definition this work uses.

\subsection{Context-Aware System}
Like context, context-awareness has had several definitions over the years. The first definition was done by \cite{schilit1994disseminating}, and it restricted to applications informed about context and applications that adapt themselves to context. Later on, synonyms have been used to define a context-aware system: reactive \cite{cooperstock1995evolution}, responsive \cite{elrod1993responsive}, situated \cite{hull1997towards}, context-sensitive  \cite{rekimoto1998augment} and environment-directed \cite{fickas1997software}. All these definitions refer to either using context, adapting to context, or both.

The definition used in this work is provided by \cite{dey2000providing}: ''A system is context-aware if it uses context to provide relevant information and/or services to the user, where relevancy depends on the user's task.''

\subsection{How to represent Context}
According to \cite{dey2000providing}, context-aware applications deal with the who's, where's, when's and what's (the activities that are occurring) of entities, and interpret this information to define why a situation is occurring. The designer of the application must decide what to do with the information. Once we have the information available, either through automated sensors or through user's interference, we need to represent it in a way a machine can process and store it.

\cite{baldauf2007survey} presents and defines the most relevant context modeling options: key-value, markup scheme, graphical, object oriented, logic based and ontology based models. As this work is developed over the same base as \cite{crippa2010}, it uses the same context representation model: a markup scheme variation, ContextML \cite{knappmeyer2010contextml}. The network nature of the messages (HTTP messages, in this case) facilitates textual, non graphic model. This is the common approach used in SOAs. 

\subsection{ContextML}

ContextML is an XML-based representation schema for context information, where it is categorized into scopes and related to different types of entities. It is designed to be used with REST-based communication between the framework components \cite{knappmeyer2010contextml}. It was created within a project called C-CAST \cite{ccast}, a collaborative work of many companies, research centers and universities, and its main objective is to evolve mobile multimedia multicasting to exploit the increasing integration of mobile devices with our everyday physical world and environment \cite{crippa2010}. The architecture of the system is based on this project.

The system consists on three core components: Context Provider, Context Consumer and Context Broker. They are briefly described below, based on definitions made by \cite{knappmeyer2010contextml}. 



\subsubsection{Context Representation}



\subsubsection{Context Provider}
A Context Provider (CxP) provides context information of a certain type, e.g. weather, location, activity, etc. It gathers data from sensors, network, user interactions, or other sources. A CxP is specialized in a specific domain of context information (location, weather etc).


\subsubsection{Context Consumer}
A Context Consumer (CxC) queries for and uses context data, e.g. is a context-aware application. A CxC can retrieve context information asynchronously through a subscription method, or by a synchronous method where it requests the Broker for a specific information or for a particular Provider interface, to query the Provider directly.

\subsubsection{Context Broker}
A Context Broker (CxB) is the central component of the architecture, and is the focus of this work. It handles and aggregates context information, and is an interface between the other architecture components. The CxB allows CxCs to subscribe to context information, and CxPs to provide this information. It also provides a lookup service, where the CxCs can query the CxB for CxPs that have a particular capability, depending on the CxC’s interest.

\subsubsection{Entity and Scope}
An entity is the subject of interest which context data refers to, and it is composed of two parts: a type and an identifier. The type refers to the category of the entity: username for human users, imei for mobile devices, room for a room with sensors, etc. The identifier specifies a particular item within a set of entities of the same type.

A scope is a set of closely related context parameters. Every context parameter has a name and belongs to only one scope. The parameters of a scope can only be requested, updated, provided and store at the same time, making the data always consistent. For example, a scope \textit{position} has latitude, longitude and accuracy attributes; any operation on this scope is performed on all these attributes: if the latitude is updated, so is the longitude and accuracy, what is correct, because otherwise it would not make sense. Entity-scope association is illustrated in Figure \ref{fig:entityscope}. \par

\begin{figure}[H]
	\centering
	\includegraphics[scale=1]{entityscope.pdf}
	\caption{Entity and Scope relationship. Source: \cite{knappmeyer2010contextml}}
	\label{fig:entityscope}
	
\end{figure}

\subsubsection{ContextML Messages}
Within the architecture of the system, context is registered, updated and queried following a set of pre-defined ContextML messages, that follow the proposed by \cite{knappmeyer2010contextml}. The possible ContextML messages and their uses are shown below, as well as a simple example of each one.

\begin{description}
\item[Advertisement Message]\hfill \\
An Advertisement Message is used by the Context Provider to register its capabilities to the broker. It informs the CxP’s access url (urlRoot), what scopes it supports (scopes), its identifier (name), and optional information about the CxP’s location. An example of a Advertisement Message can be seen in Figure \ref{fig:advertisement}.

\begin{figure}[H]
	\centering
	\includegraphics[scale=0.5]{advertisement.png}
	\caption{Provider Advertisement example message}
	\label{fig:advertisement}
	
\end{figure}

\item[CxP Lookup Message]\hfill \\
When a CxC wants to know where it can find a specific scope, it can query the CxB about which of the registered Providers has the desired information. The Broker replies with a ContextML message, describing the Providers that match with the data required by the CxC. An example can be seen in Figure \ref{fig:providers}.

\begin{figure}[H]
	\centering
	\includegraphics[scale=0.5]{providers.png}
	\caption{Providers Lookup example message}
	\label{fig:providers}
	
\end{figure}

\item[ACK Message]\hfill \\
Acknowledgement is a control message that confirms the execution of various management actions (e.g. advertisement, context update). Each ACK message contains the status of the operation, the HTTP response code, and the identification of the method it corresponds. It also has optional fields to inform scope and entity information. An example is shown in Figure \ref{fig:acknack}.

\begin{figure}[h]
	\centering
	\includegraphics[scale=0.5]{acknack.png}
	\caption{ACK and NACK example messages}
	\label{fig:acknack}
	
\end{figure}


\item[Context Representation Message]\hfill \\
When a Consumer requests or subscribes to a context scope, it receives a ContextML message with the element \textit{‘ctxEl’}, when the information queried is available. \textit{ctxEl} contains information of the provider that has the context queried (contextProvider), the entity and scope it is related to, and the context data in the \textit{dataPart} element. \textit{par}, \textit{parS} and \textit{parA} are constructors to store name-value pairs and attribute collections (structs and arrays) respectively. Every context information that is exchanged is tagged with a \textit{timestamp} (time of its generation) and an expiration time \textit{expires} (validity of the context information), after which the information is considered invalid. An example of a Context Representation Message is shown in Figure \ref{fig:ctxels}.

\begin{figure}[h]
	\centering
	\includegraphics[scale=0.5]{ctxels.png}
	\caption{Context Element example messages (update)}
	\label{fig:ctxels}
	
\end{figure}


\end{description}


\section{Fault Tolerance}
\label{sec:fault_tolerance}
\cite{arsanjani2004service} \cite{avivzienis2004basic}   \cite{barreradesign}  \cite{skeen1981nonblocking} 
\subsection{Dependable System}
\subsection{High Availability}
\subsubsection{Uses in real world}
\subsubsection{Problems involving High Availability}


\chapter{Design and Implementation}
\label{chap:implementation}
Bla.

\section{Design and Implementation of the Context Broker}
\label{sec:broker}
This work first implements a regular Context Broker, with no fault tolerance resources. Then, it proposes a strategy to give the Broker High Availability function.
 
\subsection{Platform Choice}
The programming language chosen for the development of this work was Python. Python is a powerful and easy to learn modern programming language \cite{python}. It was chosen because it represents a challenge, and to show that the system is independent of the programmed language, i.e. different applications developed on different programming languages can interact with each other in the architecture, the messages exchanged are what matters. The Python IDE \cite{pycharm} was used, along with GitHub for version control \cite{github}.

The system was implemented over a HTTP REST (Representational State Transfer) Interface. A REST Interface is \cite{fielding2002principled}. For the RESTful implementation, Python Flask framework was used \cite{flask}.

For data persistence, MongoDB was used. MongoDB is a NoSQL 

\subsection{System Architecture}
In Figure \ref{fig:diagram} an overall diagram of the system is illustrated. Each node is a component of the system, and the arrows represent the interaction between them.

\begin{figure}[h]
	\centering
	\includegraphics[scale=0.5]{diagram.png}
	\caption{Architecture diagram}
	\label{fig:diagram}
	
\end{figure}


\subsection{Broker Interfaces}
The Context Broker implements several interfaces for communication with the other system components. This section presents each interface and the way they were implemented: what they expect as input (HTTP request from Consumer or Provider), the action they perform, and what they provide as output (response to the Consumer or Provider).


\subsubsection{Advertisement}
\begin{itemize}
	\item[Input:] an Advertisement ContextML message, with Provider information
	
	\item[Action:] registers the Provider within the Broker
	
	\item[Output:] responds the Provider with a ACK or NACK ContextML message, informing success or error, with the corresponding error message.
\end{itemize}

\subsubsection{Update}
\begin{itemize}
	\item[Input:] a ctxEl ContextML Message, with context information to be registered in the Broker
	
	\item[Action:] registers in the Registry Table the context information, with its \textit{contextProvider}, \textit{scope} and \textit{entity} information, \textit{timestamp} and expiration date (\textit{expires}) of the information. It also checks if a \textbf{Subscription} exists for the updated information, sending it to the Consumer \textit{callbackUrl}, when applied.
	
	\item[Output:] responds the Provider with a ACK or NACK ContextML message, informing success or error, with the corresponding error message.
\end{itemize}

\subsubsection{Get Providers}
\begin{itemize}
	\item[Input:] \textit{scope} (mandatory) and \textit{entity type} (optional) arguments in the URL
	
	\item[Action:] looks for registered Providers that provide information matching the arguments given
	
	\item[Output:] responds the Consumer with Providers Lookup ContextML message, containing a list of the providers that match the requested arguments
\end{itemize}

\subsubsection{Get Context}
\begin{itemize}
	\item[Input:] from the Consumer, arguments \textit{scope} and \textit{entity} in the URL
	
	\item[Action:] looks for the latest Context information in the registry that matches the entity and scope received
	
	\item[Output:] responds the Consumer with a ctxEls ContextML message, containing the , or with a NACK ContextML message, informing the error.
	
\end{itemize}
* As seen in Figure \ref{fig:diagram}, there can also exist a direct \textit{GetContext} request from the Consumer to the Provider, thus not involving the Broker. This can be done by the Consumer asking the Broker for a providers list regarding a certain \textit{scope}, and then asking it directly for the desired context information. 

\subsubsection{Subscribe}
\begin{itemize}
	\item[Input:] arguments as follows:  \textit{callbackUrl}, with the URL to where the Broker sends the content it is subscribed to; \textit{scope} and \textit{entity}, with corresponding information the consumer wants to subscribe to; and \textit{minutes}, with the amount of time, in minutes, that the subscription is valid.
	
	\item[Action:] registers the subscription
	
	\item[Output:] responds the Provider with a ACK or NACK ContextML message, informing success or error, with the corresponding error message.
\end{itemize}

\section{Introducing High Availability Technique}
\label{sec:ha_broker}

\subsection{Objective}

\subsection{Design}

\subsection{Protocol created}
Using \cite{protobuf}
\chapter{Tests}
\label{chap:tests}
Bla.

 
Figure \ref{fig:broker}. \par

\begin{figure}[H]
	\centering
	\includegraphics[scale=0.2]{broker.png}
	\caption{Broker Interface}
	\label{fig:broker}
	
\end{figure}

Figure \ref{fig:provider}. \par

\begin{figure}[H]
	\centering
	\includegraphics[scale=0.2]{provider.png}
	\caption{Provider Interface}
	\label{fig:provider}
	
\end{figure}

Figure \ref{fig:consumer}. \par

\begin{figure}[H]
	\centering
	\includegraphics[scale=0.2]{consumer.png}
	\caption{Consumer Interface}
	\label{fig:consumer}
	
\end{figure}

Figure \ref{fig:brokerprovs}. \par

\begin{figure}[H]
	\centering
	\includegraphics[scale=0.2]{brokerprovs.png}
	\caption{Providers Table}
	\label{fig:brokerprovs}
	
\end{figure}

Figure \ref{fig:registries}. \par

\begin{figure}[H]
	\centering
	\includegraphics[scale=0.2]{registries.png}
	\caption{Registry Table - Context Information}
	\label{fig:registries}
	
\end{figure}

Figure \ref{fig:subscriptions}. \par

\begin{figure}[H]
	\centering
	\includegraphics[scale=0.2]{subscriptions.png}
	\caption{Subscriptions Table}
	\label{fig:subscriptions}
	
\end{figure}

Figure \ref{fig:log}. \par

\begin{figure}[H]
	\centering
	\includegraphics[scale=0.2]{log.png}
	\caption{Broker Log}
	\label{fig:log}
	
\end{figure}

 
\chapter{Conclusion}
\label{chap:conclusion}
This work introduces context and context-awareness concepts, for a better understanding of context-aware computing and how it is used. Concepts of Dependability and High Availability were also presented, to demonstrate how these are important in research and applicability. For the design of the high availability proposal in this work, Cluster-like systems behavior was part of the inspiration to find a design for the highly available Context Broker. The implementation of the Broker was presented, as with its interfaces and UML use cases description, for a better understanding of the functionality of the system. Simple tests were made, only to certify the well operating of the Context Broker. 
In addition of that, a protocol for integrating high availability to the Context Broker was proposed. This protocol is based on existing nonblocking commit protocols, and given the description of it, is not very difficult to prototype.

The goals of this work were successfully achieved. A Context Broker was developed following a previous definition, but with a different programming language (Python) and data storage mechanism (MongoDB). A different approach from \cite{crippa2010}, but resulting in the same system from the client's (Providers and Consumers) point of view. The proposed protocol was derived from existing solutions, giving it some basis on its viability. A prototype is the natural next step.




\section{Future Work}
This work initiates a study on the development of a Dependable Broker, and naturally some future work is proposed.

\begin{description}
	\item[Prototype for the proposed solution:] development of a prototype that implements the idea presented in this work and confirms its viability, as well as looking for an optimal time-out value for the messages between Brokers
	
	\item[Election of Broker] study of solutions to the election of a Broker to respond the request for another Broker that has failed, using or adapting existing protocols, for a better fit in the Broker system
	
	\item[Recovery algorithms] study of recovery algorithms, and how to implement them on the current Broker system
	
	\item[Security concerns:] use of HTTPS protocols, authentication method between Provider and Consumer through the Broker, types of data: public and private, requiring some kind of authentication for a Consumer to access this information
	
	\item[Scalability:] make possible the addition and removal of Brokers to the Broker system, without the need to stop or restart it, dealing with membership and consensus problems
	
	\item[Selective High Availability:] define classes of context data, regarding the need of high availability of those, making some type of data, e.g. location information with small time separation, not always highly available, what can spare some message exchanging, making the system faster
	
	\item[Dependability:] try to attend other Dependability characteristics, e.g. safety and correctness of service

\end{description}

 


% e aqui vai a parte principal
% 
% \chapter{Estado da arte}
% \chapter{Mais estado da arte}
% \chapter{A minha contribuição}
% \chapter{Prova de que a minha contribuição é válida}
% \chapter{Conclusão}

% referências
% aqui será usado o environment padrao `thebibliography'; porém, sugere-se
% seriamente o uso de BibTeX e do estilo abnt.bst (veja na página do
% UTUG)
% 
% observe também o estilo meio estranho de alguns labels; isso é
% devido ao uso do pacote `natbib', que permite fazer citações de
% autores, ano, e diversas combinações desses

\bibliographystyle{abntex2-alf}
\bibliography{biblio}

\end{document}
